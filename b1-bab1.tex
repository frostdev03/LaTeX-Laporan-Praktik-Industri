%==================================================================
% Ini adalah bab 1
% Silahkan edit sesuai kebutuhan, baik menambah atau mengurangi \section, \subsection
%==================================================================

\chapter[PENDAHULUAN]{\\ PENDAHULUAN}

\section{Latar Belakang}
Latar belakang pelaksanaan praktik industri atau magang bagi mahasiswa sarjana memiliki signifikansi yang besar dalam mengembangkan keterampilan praktis dan pengetahuan yang relevan dengan dunia kerja. Praktik industri adalah wadah penting bagi mahasiswa untuk mengaplikasikan teori yang mereka pelajari di kampus ke dalam konteks pekerjaan sehari-hari. Dengan melibatkan mahasiswa dalam pengalaman praktis di lingkungan kerja nyata, mereka dapat memperoleh pemahaman mendalam tentang dinamika industri yang tidak dapat diperoleh melalui pembelajaran klasikal.

Selain itu, praktik industri juga membuka peluang bagi mahasiswa untuk membangun jaringan profesional, mengembangkan soft skills, dan mengeksplorasi berbagai potensi karir. Latar belakang praktik industri ini mencerminkan komitmen institusi pendidikan terhadap pendekatan holistik dalam mempersiapkan mahasiswa untuk menghadapi tantangan dunia kerja. Dengan memahami pentingnya latar belakang ini, praktik industri dapat dirancang secara lebih kontekstual dan sesuai dengan perkembangan terkini di industri, memberikan kontribusi positif bagi perkembangan karir dan pertumbuhan pribadi mahasiswa sarjana.

\section{Tujuan \tipe}
Tujuan dari praktik industri atau magang adalah memberikan pengalaman praktis kepada mahasiswa sarjana agar mereka dapat mengintegrasikan pengetahuan teoritis dengan aplikasi dunia nyata. Melalui partisipasi dalam lingkungan kerja, mahasiswa dapat mengembangkan keterampilan teknis dan non-teknis yang diperlukan dalam karir mereka. Praktik industri bertujuan untuk memberikan pemahaman mendalam tentang operasi industri, proses kerja, dan dinamika tim.

Selain itu, tujuan praktik industri juga termasuk memfasilitasi pengembangan jaringan profesional, memperluas wawasan mahasiswa terhadap berbagai aspek industri, dan meningkatkan kesiapan mereka menghadapi tantangan dunia kerja. Program ini berupaya membekali mahasiswa dengan keterampilan adaptasi, pemecahan masalah, serta kemampuan berkomunikasi dan bekerja sama dalam tim.

Selama praktik industri, mahasiswa diharapkan dapat mengidentifikasi potensi karir, menyesuaikan diri dengan budaya kerja, dan membangun kompetensi yang relevan dengan bidang studi mereka. Dengan demikian, tujuan praktik industri adalah menciptakan lulusan yang tidak hanya memiliki pengetahuan akademis, tetapi juga keterampilan praktis yang dibutuhkan oleh industri, sehingga mereka dapat menjadi kontributor yang berdaya saing dalam pasar kerja global.

\subsection{Tujuan Umum}

Tujuan umum dari pelaksanaan praktik industri atau magang adalah meningkatkan kesiapan dan kompetensi mahasiswa sarjana untuk memasuki dunia kerja dengan pengetahuan dan keterampilan yang relevan. Program ini bertujuan untuk menyediakan pengalaman praktis yang mendalam, memungkinkan mahasiswa mengintegrasikan teori dengan aplikasi praktis, serta memperluas wawasan mereka terhadap berbagai aspek industri.

\subsection{Tujuan Khusus}

Tujuan khusus dari \tipe adalah sebagai berikut:

\begin{packed_enum}
	\item Pengembangan Keterampilan Teknis: Memberikan mahasiswa kesempatan untuk mengasah keterampilan teknis yang diperlukan dalam bidang studi mereka melalui penerapan konsep-konsep teoritis dalam situasi kerja nyata.
	\item Pengembangan Keterampilan Soft Skills: Meningkatkan keterampilan interpersonal, komunikasi, kepemimpinan, dan kerja sama tim, yang merupakan aspek penting dalam lingkungan kerja.
	\item Pengenalan pada Budaya Kerja: Memungkinkan mahasiswa memahami budaya perusahaan, norma-norma, dan nilai-nilai yang mempengaruhi dinamika organisasi.
	\item Pengembangan Jaringan Profesional: Membantu mahasiswa membangun hubungan profesional dengan praktisi industri, sesama mahasiswa, dan pemimpin perusahaan, yang dapat mendukung perkembangan karir mereka di masa depan.
	\item Pemahaman Proses Bisnis: Memberikan pemahaman mendalam tentang proses bisnis dan operasional perusahaan, yang dapat menjadi dasar bagi pemikiran analitis dan pengambilan keputusan.
	\item Pengembangan Karir: Membantu mahasiswa mengidentifikasi minat karir, mengembangkan rencana karir, dan mempersiapkan mereka untuk sukses dalam mencari pekerjaan setelah lulus.
\end{packed_enum}

\section{Manfaat \tipe}
Terdapat 3 kategori dari manfaat pelaksanaan \tipe. Berikut adalah penjelasan manfaat bagi mahasiswa, universitas dan industri.

\subsection{Bagi Mahasiswa}

\begin{packed_enum}
	\item Pengalaman Praktis: Memberikan mahasiswa kesempatan untuk menerapkan pengetahuan teoritis dalam konteks pekerjaan nyata, mengasah keterampilan praktis yang diperlukan dalam karir masa depan.
	\item Pengembangan Keterampilan Soft Skills: Memperkuat keterampilan interpersonal, komunikasi, kepemimpinan, dan kerja sama tim, meningkatkan daya saing di pasar kerja.
	\item Pengenalan pada Dunia Kerja: Menyediakan wawasan mendalam terhadap budaya kerja, etika profesional, dan tuntutan industri, membantu mahasiswa mengadaptasi diri dengan cepat setelah lulus.
	\item Pengembangan Jaringan Profesional: Membuka peluang untuk membangun hubungan dengan profesional industri, potensial mentor, dan sesama mahasiswa, membantu dalam membangun jaringan karir.
\end{packed_enum}

\subsection{Bagi Universitas}

\begin{packed_enum}
	\item Peningkatan Kualitas Pendidikan: Mendukung pendidikan holistik dengan menyediakan pengalaman praktis yang melengkapi pembelajaran klasikal, mempersiapkan mahasiswa untuk sukses di dunia kerja.
	\item Hubungan Industri-Akademis: Membangun kemitraan erat antara universitas dan industri, memastikan relevansi kurikulum dengan kebutuhan pasar kerja.
	\item Peningkatan Citra Institusi: Menunjukkan komitmen universitas terhadap penyediaan lulusan yang siap kerja, meningkatkan reputasi institusi di mata pemangku kepentingan.
\end{packed_enum}

\subsection{Bagi Industri}

\begin{packed_enum}
	\item Sumber Talenta Unggul: Menyediakan akses langsung ke calon karyawan berkualitas dengan keterampilan yang sesuai dengan kebutuhan industri.
	\item Inovasi dan Ide Baru: Mendapatkan perspektif segar dari mahasiswa, memperkaya inovasi dan membantu perusahaan tetap relevan di pasar yang terus berubah.
	\item Membangun Hubungan Jangka Panjang: Memberikan peluang untuk membangun koneksi dengan institusi pendidikan, menciptakan aliran bakat dan kolaborasi jangka panjang.
\end{packed_enum}