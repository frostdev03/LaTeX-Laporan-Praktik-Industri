%==================================================================
% Ini adalah bab 2
% Silahkan edit sesuai kebutuhan, baik menambah atau mengurangi \section, \subsection
%==================================================================

\chapter[PROFIL PERUSAHAAN]{\\ PROFIL PERUSAHAAN}

\section{Sejarah \perusahaan}
Dalam sub bab ini, mahasiswa harus menyajikan gambaran umum tentang latar belakang perusahaan tempat mereka melakukan praktik. Informasi yang perlu disertakan meliputi:

\begin{packed_enum}
    \item Tahun berdirinya perusahaan dan pendirinya.
    \item Perkembangan perusahaan dari awal berdiri hingga saat ini, termasuk pencapaian-pencapaian penting.
    \item Perubahan signifikan yang pernah terjadi, seperti merger, akuisisi, atau ekspansi ke pasar baru.
    \item Posisi perusahaan dalam industri elektronika, baik secara nasional maupun internasional jika relevan.
\end{packed_enum}

Penulisan harus bersifat ringkas dan informatif, memberikan pembaca pemahaman yang baik tentang perjalanan perusahaan.

\section{Visi dan Misi Perusahaan}
Pada bagian ini, mahasiswa harus menjelaskan visi dan misi perusahaan. Visi adalah pernyataan tentang cita-cita jangka panjang perusahaan, sedangkan misi adalah langkah-langkah strategis yang diambil untuk mencapai visi tersebut. Mahasiswa harus:

\begin{packed_enum}
    \item Menuliskan visi perusahaan dengan jelas.
    \item Menuliskan misi perusahaan dan menjelaskan bagaimana misi tersebut mendukung pencapaian visi.
    \item Mengaitkan visi dan misi dengan aktivitas yang dilakukan selama praktik, jika memungkinkan.
\end{packed_enum}
Pastikan untuk menuliskan ini sesuai dengan yang tertera dalam dokumen resmi perusahaan atau informasi yang diberikan oleh pihak perusahaan.

\section{Struktur Organisasi}
Sub bab ini bertujuan untuk menjelaskan bagaimana perusahaan diorganisir. Mahasiswa harus:

\begin{packed_enum}
    \item Menyajikan bagan struktur organisasi perusahaan, jika tersedia. Jika tidak, deskripsikan struktur organisasi secara naratif.
    \item Menjelaskan peran dan tanggung jawab dari masing-masing bagian atau departemen dalam perusahaan.
    \item Menyebutkan posisi atau departemen tempat mahasiswa melakukan praktik dan menjelaskan bagaimana posisi tersebut berkontribusi terhadap keseluruhan operasi perusahaan.
\end{packed_enum}

Struktur organisasi membantu pembaca memahami alur kerja dan hubungan antar departemen di perusahaan.

\section{Bidang Usaha dan Produk/Jasa yang Dihasilkan}
    JALA berfous pada memajukan industri akuakultur terkhusus udang. JALA berkontribusi secara end-to-end,  
Di bagian ini, mahasiswa harus menjelaskan fokus bisnis perusahaan dan produk atau jasa yang dihasilkan. Informasi yang perlu disertakan meliputi:

\begin{packed_enum}
    \item Deskripsi bidang usaha utama perusahaan, misalnya manufaktur komponen elektronik, pengembangan perangkat lunak, atau layanan konsultasi teknologi.
    \item Daftar produk atau jasa utama yang ditawarkan oleh perusahaan, termasuk spesifikasi teknis atau fitur utama jika relevan.
    \item Pasar atau segmen pelanggan yang menjadi target perusahaan.
\end{packed_enum}

Penulisan harus jelas dan detail, memberikan pembaca gambaran yang baik tentang apa yang dilakukan perusahaan dan produk atau jasa yang dihasilkannya.