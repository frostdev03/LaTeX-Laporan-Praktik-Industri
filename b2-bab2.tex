%==================================================================
% Ini adalah bab 2
% Silahkan edit sesuai kebutuhan, baik menambah atau mengurangi \section, \subsection
%==================================================================

\chapter[RENCANA KEGIATAN PRAKTIK INDUSTRI]{\\ RENCANA KEGIATAN PRAKTIK INDUSTRI}

\section{Jadwal Kegiatan}
\begin{table}[h]
	\centering
	\renewcommand{\arraystretch}{1}
	\setlength{\tabcolsep}{1.5pt}
	\small
	\begin{tabular}{|c|l|c|l|c|l|c|l|c|l|c|l|c|l|c|l|c|l|c|l|c|l|c|l|c|l|c|l|c|l|c|l|c|l|c|l|c|l|c|l|c|l|c|l|}        \hline
		\multirow{2}{*}{NO} & \multirow{2}{*}{KEGIATAN} & \multicolumn{4}{c|}{FEB} & \multicolumn{4}{c|}{MAR} & \multicolumn{4}{c|}{APR} & \multicolumn{4}{c|}{MEI} & \multicolumn{4}{c|}{JUN} & \multicolumn{4}{c|}{JUL} & \multicolumn{1}{c|}{AGS} \\ \cline{3-26}
		& & 1 & 2 & 3 & 4 & 1 & 2 & 3 & 4 & 1 & 2 & 3 & 4 & 1 & 2 & 3 & 4 & 1 & 2 & 3 & 4 & 1 & 2 & 3 & 4 & 1 \\
		\hline
		1 & Pengenalan Industri & X &  &  &  &  &  &  &  &  &  &  &  &  &  &  &  &  &  &  &  &  &  &  &  & \\
		\hline
		2 & Mempelajari Tugas & X & X &  &  &  &  &  &  &  &  &  &  &  &  &  &  &  &  &  &  &  &  &  &  & \\
		\hline
		3 & Pelaksanaan Praktik Industri &  &  & X & X & X & X & X & X & X & X & X & X & X & X & X & X & X & X & X & X & X & X & X & X & X \\
		\hline
		4 & Pengambilan Data &  &  &  & X & X & X & X & X & X & X & X & X & X & X & X & X & X & X & X & X & X & X & X & X &  \\
		\hline
		5 & Penyusunan Laporan &  &  &  &  &  &  &  & X & X & X & X & X & X & X & X & X & X & X & X & X & X & X & X & X &  \\
		\hline
		6 & Evaluasi &  &  &  &  &  &  &  &  &  &  &  &  &  &  &  &  &  &  &  &  &  &  &  & X & X \\
		\hline
	\end{tabular}
	\caption{Jadwal Kegiatan Praktik Industri}
\end{table}



\section{Deskripsi Kegiatan}

\subsection{Pengenalan Industri}
Pada tahap awal praktik industri, peserta dikenalkan dengan lingkungan kerja, struktur organisasi, budaya perusahaan, serta standar operasional yang berlaku. Peserta juga mendapatkan pemahaman tentang visi dan misi industri tempat peserta melakukan praktik. Selain itu, peserta akan berkenalan dengan mentor atau supervisor yang akan membimbing peserta selama praktik berlangsung.

\subsection{Mempelajari Tugas}
Setelah memahami lingkungan industri, peserta mulai mempelajari tugas yang diberikan. Peserta harus memahami peran dan tanggung jawab yang harus dijalankan selama praktik. Selain itu, peserta akan mengenali alat atau perangkat lunak yang digunakan dalam pekerjaan serta memahami standar keselamatan kerja dan regulasi yang berlaku di tempat praktik.

\subsection{Pelaksanaan Praktik Industri}
Pada tahap ini, peserta mulai mengerjakan tugas sesuai dengan bidangnya. Peserta menerapkan teori yang telah dipelajari dalam lingkungan industri nyata serta berinteraksi dengan rekan kerja dan tim proyek. Dalam proses ini, peserta berkesempatan untuk mengembangkan keterampilan teknis dan profesional yang dibutuhkan di dunia kerja.

\subsection{Pengambilan Data}
Selama praktik berlangsung, peserta diwajibkan untuk mengumpulkan data yang relevan dari pekerjaan yang peserta lakukan. Data tersebut mencakup pencatatan proses kerja, kendala yang dihadapi, serta solusi yang diterapkan. Selain itu, peserta juga mendokumentasikan hasil pekerjaan dalam bentuk catatan, foto, atau laporan harian sebagai bahan penyusunan laporan akhir.

\subsection{Penyusunan Laporan}
Setelah data terkumpul, peserta menyusun laporan praktik industri berdasarkan pengalaman yang telah dijalani. Laporan ini harus mencakup hasil yang diperoleh, analisis praktik yang dilakukan, serta evaluasi terhadap pencapaian dan tantangan selama praktik berlangsung. Penulisan laporan mengikuti format yang telah ditentukan oleh institusi pendidikan atau industri tempat praktik dilakukan.

\subsection{Evaluasi}
Tahap akhir dari praktik industri adalah evaluasi, di mana peserta menyampaikan hasil praktik dalam bentuk presentasi atau diskusi. Peserta akan menerima umpan balik dari mentor atau supervisor mengenai kinerja yang telah ditunjukkan selama praktik. Evaluasi ini juga menjadi kesempatan bagi peserta untuk merefleksikan pengalaman yang didapat dan memberikan rekomendasi untuk peningkatan keterampilan di masa depan.

\section{Lokasi dan Fasilitas}
Peserta program paid internship di JALA Tech Engineerium diwajibkan untuk menjalankan kerja penuh secara Work From Office (WFO) selama tiga bulan pertama. Lokasi magang berada di JALA Tech Engineerium, yang terletak di Jl. Perkutut Gg. Kenari Indah I No.8, Malangrejo, Wedomartani, Kec. Ngemplak, Kabupaten Sleman, Daerah Istimewa Yogyakarta 55584. Dalam periode ini, peserta akan bekerja selama lima hari dalam seminggu, dengan durasi kerja enam jam per hari dalam sistem flexible working hour, yakni antara pukul 09.00 hingga 15.00.

Selama mengikuti program ini, peserta akan mendapatkan berbagai fasilitas dan tunjangan, termasuk sertifikat magang sebagai bukti pengalaman kerja, merchandise eksklusif dari JALA Tech, serta uang saku senilai Rp. 1.250.000 per bulan untuk tiga bulan pertama.

\subsection{Profil Perusahaan \perusahaan}
JALA adalah perusahaan yang berfokus pada inovasi di industri budidaya udang dengan memanfaatkan teknologi dan analisis data. JALA hadir untuk membantu petambak udang mengatasi berbagai tantangan dalam budidaya dengan menyediakan layanan yang terintegrasi dengan teknologi, analisis akuakultur, serta informasi terbaru seputar industri perikanan. Dengan solusi berbasis data, JALA membantu petambak meningkatkan hasil panen, memitigasi risiko kegagalan, dan menerapkan sistem budidaya yang berkelanjutan.

Sebagai pemimpin dalam revolusi biru di industri tambak udang, JALA memiliki visi untuk menjadi pelopor dalam penerapan big data guna meningkatkan produktivitas perikanan. Misinya adalah meningkatkan produksi dan efisiensi budidaya melalui pengambilan keputusan berbasis data, sehingga petambak dapat mengoptimalkan hasil panen dengan cara yang lebih efektif dan berkelanjutan.

\subsection{Peralatan yang digunakan}
1. Laptop/PC

2. Device JALA Baruno

3. Software (Visual Studio Code, STM32cube IDE dan Arduino IDE)



\section{Konversi Mata Kuliah}
1. Praktik Industri Terbimbing (8 SKS).
2. Praktik Industri Mandiri (8 SKS).
Total konversi dari kegiatan praktik industri ini adalah 16 SKS karena telah memenuhi durasi yang ditentukan yaitu 6 bulan.

\section{Evaluasi dan Pelaporan}
Selama praktik industri peserta membuat catatan harian dan membuat laporan akhir yang nanti akan dijadikan sebagai acuan dosen dalam menilai kegiatan tersebut. Dengan catatan harian dan laporan akhir akan memberikan pelaporan yang akurat dan mempermudah dalam evaluasi.